\documentclass[beamer]{standalone}

\usepackage{tikz}
\usepackage{pgfplots}

\usetikzlibrary{positioning, shadows, fit, arrows.meta, backgrounds, overlay-beamer-styles}


\begin{document}
\pgfplotsset{delpoint/.style={
			scatter,
			thick,
			mark=o,
			mark size=5pt,
			only marks,
			scatter,
		}}

% Define style for line plots
\pgfplotsset{delline/.style={
			mesh,
			thick,
			samples=50
		}}

\newcommand{\funcDel}[1]{100 * exp(-3 * #1 / 100)}
\begin{tikzpicture}
	\pgfmathsetmacro{\pointAx}{0}
	\pgfmathsetmacro{\pointBx}{23}
	\pgfmathsetmacro{\pointCx}{100}

	\pgfmathsetmacro{\pointAy}{\funcDel{\pointAx}}
	\pgfmathsetmacro{\pointBy}{\funcDel{\pointBx}}
	\pgfmathsetmacro{\pointCy}{\funcDel{\pointCx}}
	\begin{axis}[
			title={Deletion},
			xlabel={Perturbation Amount},
			ylabel={Predicted Class Probability},
			axis lines=left,
			xmin=-10, xmax=110,
			ymin=-10, ymax=110,
			xtick={0,100},
			ytick={0,100},
			legend pos=north west,
			%ymajorgrids=true,
			%grid style=dashed,
			%yticklabel style={rotate=90}, % Rotate y-axis labels by 45 degrees
			xticklabel=\pgfmathprintnumber{\tick}\%, % Add % sign to x-axis labels
			yticklabel=\pgfmathprintnumber{\tick}\%, % Add % sign to y-axis labels
			colormap={coolwarm}{
					rgb(0cm)=(0,0,1);
					rgb(1cm)=(1,0,0);
				},
			scatter src=y,
		]

		% Define a variable
		%\pgfmathsetmacro{\phi}{3}

		% Define a function
		%\pgfmathsetmacro{\f}{100 * exp(-\phi * x)}

		\only<1>{
			\addplot[delpoint]
			coordinates {
					(\pointAx, \pointAy)
				};
		}
		\only<2>{
			\addplot[delpoint]
			coordinates {
					(\pointAx, \pointAy)
					(\pointBx, \pointBy)
				};
			\addplot[delline, domain=0:23] {\funcDel{x}};
		}
		\only<3>{
			\addplot[delpoint]
			coordinates {
					(\pointAx, \pointAy)
					(\pointBx, \pointBy)
					(\pointCx, \pointCy)
				};
			\addplot[delline, domain=0:100] {\funcDel{x}};
		}


		\node at (axis cs: \pointAx,\pointAy) (pointA) {};
		\node at (axis cs: \pointBx,\pointBy) (pointB) {};
		\node at (axis cs: \pointCx,\pointCy) (pointC) {};

	\end{axis}

	\visible<1->{
		\node[draw, rectangle] at (pointA -| 10, 0) (inputA) {};
		%\node[draw, rectangle,right=5cm of pointA] (inputA) {};
		\draw[->] (inputA) -- (pointA);
	}

	\visible<2->{
		\node[draw, rectangle] at (pointB -| 10, 0) (inputB) {};
		\draw[->] (inputB) -- (pointB);
	}

	\visible<3->{
		\node[draw, rectangle] at (pointC -| 10, 0) (inputC) {};
		\draw[->] (inputC) -- (pointC);
	}

\end{tikzpicture}
\end{document}